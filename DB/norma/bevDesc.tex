A norma egy vektortér elemeihez rendelt mennyiség, mely: 
\centerline{\mcode{$||0||=0$,\text{\ egyébként: \ }$||x||>0$}}
\centerline{\mcode{$||c x||=|c| ||x||$}}
\centerline{\mcode{teljesül a $\triangle$-\elen{}}}

Legismertebb vektor-normák:
\Mat{
   ||x||_p = \gzjel{|x_1|^p+\ldots|x_n|^p}^{\frac{1}{p}}\kh p\ge 1 \nus
   ||x||_{\infty}=\max_{k}(|x_k|)
}
Ez $p=2$ esetén a szokásos euklidészi-norma, $p=1$-re az ún. Manhattan-norma. 
A harmadik neve $\max$-norma.\newline
Mátrixokra a legelterjedtebb az indukált-norma használata. Kiindulva egy $||.||_v$ vektornormából, a 
\Mat{
   ||A||_v = \sup_{||x||=1} ||Ax||_v =\max_{||x||=1} ||Ax||_v
}
mennyiségről belátható, hogy norma az $n\times n$-es mátrixok vektorterén. ($n$ rögzített)
