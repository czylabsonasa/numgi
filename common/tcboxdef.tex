%https://tex.stackexchange.com/questions/254324/how-to-creat-color-boxes-as-the-picture
% FASZ = feladatszín
\definecolor{FASZ}{RGB}{160, 187, 229}
%\definecolor{LESZ}{RGB}{206, 177, 224} ez lilás
\definecolor{LESZ}{RGB}{173, 173, 235}
\definecolor{MOSZ}{RGB}{102, 209, 141}
\definecolor{LISZ}{RGB}{232, 219, 106}
%\definecolor{MASZ}{RGB}{217, 210, 221}
\definecolor{MASZ}{RGB}{210, 203, 214}
%\definecolor{BGSZ}{RGB}{224,224,224} szurke
\definecolor{BGSZ}{RGB}{220,220,220} 


\newcommand{\feher}[1]{%
\begin{tcolorbox}[colback=white]
#1
\end{tcolorbox}
}

\newcommand{\matlab}[1]{%
\feher{\lstinputlisting{DB/M/#1.m}}
}

\newcommand{\egykep}[1]{%
\begin{center}
\makebox[\textwidth]{\includegraphics[width=\textwidth]{DB/fig/#1}}
\end{center}
}

\newcommand{\zold}[1]{%
\begin{tcolorbox}[colback=green!17]
#1
\end{tcolorbox}
}

\newcommand{\szurke}[1]{%
\begin{tcolorbox}[colback=gray!10!white]
#1
\end{tcolorbox}
}

\newcommand{\szurkeM}[1]{%
\begin{tcolorbox}[colback=gray!10!white, top=-5mm, bottom=3mm]
\begin{gather*}
#1
\end{gather*}
\end{tcolorbox}
}

\newcommand{\Mat}[1]{%
\begin{tcolorbox}[colback=MASZ, top=-5mm, bottom=3mm]
\begin{gather*}
#1
\end{gather*}
\end{tcolorbox}
}

\newcommand{\MatTag}[2]{%
\begin{tcolorbox}[colback=MASZ, top=-5mm, bottom=3mm]
\begin{gather*}\label{#1}\tag{#1}
#2
\end{gather*}
\end{tcolorbox}
}


\newcommand{\LINK}[1]{%
\begin{tcolorbox}[colback=LISZ]
#1
\end{tcolorbox}
}

\newcommand{\barna}[1]{%
\begin{tcolorbox}[colback=brown!20!white]
#1
\end{tcolorbox}
}

\newcommand{\kek}[1]{%
\begin{tcolorbox}[colback=blue!20!white]
#1
\end{tcolorbox}
}

\newcommand{\Fa}[1]{%
\begin{tcolorbox}[colback=FASZ]
#1
\end{tcolorbox}
}

\newcommand{\Fnew}[0]{%
\end{tcolorbox}
\begin{tcolorbox}[colback=FASZ]
}

\newcommand{\Desc}[1]{%
\begin{tcolorbox}[colback=LESZ]
#1
\end{tcolorbox}
}

\newcommand{\Dnew}[0]{%
\end{tcolorbox}
\newpage
\begin{tcolorbox}[colback=LESZ]
}


\newcommand{\Mo}[1]{%
\begin{tcolorbox}[colback=MOSZ]
#1
\end{tcolorbox}
}


\newcommand{\Mnew}[0]{%
\end{tcolorbox}
\begin{tcolorbox}[colback=MOSZ]
}


\newcommand{\egyFa}[1]{%
\Fa{\input{#1Fa}}
\Mo{\input{#1Mo}}
}

\newcommand{\egyDesc}[1]{%
\Desc{\input{#1Desc}}
}




\definecolor{light-gray}{gray}{0.95}
\newcommand{\mcode}[1]{
   \colorbox{light-gray}{\texttt{#1}}
}



\newlength{\mlen}
\settowidth{\mlen}{12345}
\newcommand{\mtit}[1]{
   \makebox[\mlen][r]{\color{gray}#1 }
}


